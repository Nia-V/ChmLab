%=================================================================
% Modèle de devoir - Version française
%=================================================================

\documentclass[12pt,a4paper]{article}

%---------------------------------------------------------------
% PACKAGES
%---------------------------------------------------------------
\usepackage[T1]{fontenc}              % Encodage correct
\usepackage[utf8]{inputenc}           % Support UTF-8
\usepackage[french]{babel}            % Langue française
\usepackage{geometry}                 % Marges
\geometry{margin=2.0cm}
\usepackage{amsmath,amssymb,amsthm}   % Math
\usepackage{graphicx}                 % Images
\usepackage{array}                    % Tableaux avancés
\usepackage{tikz}                     % K-maps, circuits logiques
\usetikzlibrary{matrix, positioning}  % Bibliothèques TikZ nécessaires
\usepackage{caption}                  % Légendes personnalisées
\usepackage{multirow}                 % Cellules multi-lignes
\usepackage{xcolor}                   % Couleurs
\usepackage{fancyhdr}                 % En-têtes / pieds de page
\usepackage{hyperref}                 % Liens hypertextes
\usepackage{float}                    % Placement précis
\usepackage{datetime}                 % Formatage de dates
\usepackage{ctable}
\usetikzlibrary{calc}
\usepackage{ragged2e}
\usepackage{csquotes}                 % Recommandé avec biblatex
\usepackage{newtxtext,newtxmath}
\usepackage{setspace}
\usepackage{tabu}

% --- BIBLIOGRAPHIE (style APA ÉTS) ---
\usepackage[style=apa,backend=biber,sorting=nyt]{biblatex}
\setlength\bibitemsep{0.8\baselineskip}
\DeclareLanguageMapping{french}{french-apa}
\addglobalbib{sample.bib}

%---------------------------------------------------------------
% INFORMATIONS DU COURS ET DU GROUPE
%---------------------------------------------------------------
\newcommand{\codeclasse}{CHM131-}                     % Code du cours
\newcommand{\nomclasse}{Chimie et matériaux}          % Nom du cours
\newcommand{\titredevoir}{Laboratoire: Adoucissement par précipitation}
\newcommand{\groupe}{15}
\newcommand{\enseignant}{Marc Fraser}
\newcommand{\datedepot}{21 octobre 2025}
\newcommand{\Session}{A25}
\newcommand{\auteurs}{
    Ourania Voyatzis~(VOYO78260401)\\
    Kian Chowanietz~(CHOK86330401)\\
    Gubens Romulus~(ROMG77300401)\\
    Loïc Lompo~(LOMY76310501)
}

%---------------------------------------------------------------
% DÉBUT DU DOCUMENT
%---------------------------------------------------------------
\begin{document}
\onehalfspacing

\begin{titlepage}
    \RaggedLeft\auteurs\\
    \vspace*{\fill}
    \centering{
        \textbf{\titredevoir}\\[0.5em]
        \vspace*{\fill}
        \textbf{\nomclasse\ (\codeclasse\groupe)}\\[0.5em]
        \textbf{Remis à: }\enseignant \\[1em]
        \vspace*{\fill}

        \textbf{Date: } \datedepot\\
        \textbf{Date de remise:}\text{ 24 novembre 2025}
    }
\end{titlepage}



%---------------------------------------------------------------
% CONTENU
%---------------------------------------------------------------
\section*{Introduction}

\paragraph{}La présence d’ions dissous dans l’eau peut entraîner une dureté élevée, ce qui pose des problèmes pour les procédés industriels, l’entretien des équipements et la durabilité des canalisations. Une eau est dite dure lorsqu’elle contient une concentration importante d’ions alcalino-terreux, en particulier les ions calcium (\(Ca^{2+}\)), qui réagissent avec le savon et favorisent la formation de dépôts de tartre dans les installations. Dans le contexte du brassage de la bière, une dureté excessive peut altérer le fonctionnement des équipements et modifier les caractéristiques organoleptiques du produit final.

\paragraph{}Le présent laboratoire s’intéresse à l’adoucissement d’une eau souterraine provenant de Loublonnière, dont la dureté est jugée trop élevée pour une utilisation directe dans un procédé de micro-brasserie. L’objectif principal est de réduire la concentration d’ions calcium en dessous de la valeur cible, en évaluant l’efficacité d’un traitement par précipitation. Cet objectif est atteint en combinant une mesure de la dureté par titrage complexométrique et une étape de précipitation chimique suivie d’une séparation du solide formé.

\section*{Principe}

\paragraph{}Les tests préliminaires réalisés sur l’eau souterraine de Loublonnière ont montré qu’elle présente des qualités organoleptiques et chimiques favorables à la fabrication de la bière, à l’exception d’une dureté trop élevée \parencite{piotte_chm131_2025}. Une eau de consommation est considérée de bonne qualité lorsqu’elle possède un goût neutre, sans arrière-goût métallique, et qu’elle est incolore, limpide et sans odeur perceptible \parencite{mediaterre_importance_2013}. La présence de certains ions peut toutefois influencer positivement le goût final de la bière : par exemple, une eau contenant une concentration de chlorure supérieure à celle de sulfate peut \textit{« rehausser la rondeur et la douceur du malt »} \parencite{giovanisci_beginners_2022}.

\paragraph{}La dureté de l’eau est principalement causée par la présence d’ions calcium (\(Ca^{2+}\)) et magnésium (\(Mg^{2+}\)) dissous. Une concentration trop élevée en ces ions peut nuire à certains procédés industriels et altérer la saveur des boissons produites avec cette eau. \textit{« Les eaux souterraines qui ont été en contact avec des roches poreuses contenant des dépôts de minéraux tels que le calcaire ou la dolomie seront très dures »} en raison de la dissolution et de la dissociation de ces minéraux dans l'eau \parencite{mcvean_is_2019}. Cependant, cette dureté peut également être bénéfique dans une certaine mesure pour le brassage de la bière si la source est le gypse.
\paragraph{}Les effets du gypse sur le brassage sont multiples : une diminution du pH du moût, une amélioration de l’extraction du malt, une meilleure clarté ainsi qu’un certain contrôle sur l’arôme des bières fortement houblonnées. Toutefois, une élimination excessive des phosphates sous forme de phosphate de calcium insoluble peut nuire à la fermentation si le moût en devient trop pauvre \parencite[p.~562]{oliver_oxford_2013}.
\paragraph{} L'équation chimique de la dissolution du gypse dans l'eau est la suivante.
\[ CaSO_{4~~(s)} + 2H_2O \rightarrow Ca^{2+}_{~~(aq)} + SO^{2-}_{4~~(aq)} \]

\paragraph{}Dans ce laboratoire, l’adoucissement de l’eau est effectué par précipitation chimique. Un réactif, le carbonate de sodium (\(Na_2CO_3\)), contenant des ions carbonate (\(CO_3^{2-}\)) est ajouté afin de former un précipité insoluble de carbonate de calcium, éliminant ainsi une partie des ions responsables de la dureté. L’équation ionique simplifiée de cette réaction est la suivante :

\[
    \mathrm{Ca^{2+}_{~~(aq)} + CO^{2-}_{3~~(aq)} \rightarrow CaCO_{3~(s)}}
\]

\paragraph{}Le carbonate de calcium précipité est ensuite séparé mécaniquement par filtration, permettant d’obtenir une eau adoucie dont la concentration en ions calcium est réduite.


\paragraph{}Dans ce laboratoire, la quantité d'ions calcium libres dans l'eau du puits de Loublonnière a été mesurée par titrage de l'EDTA dans la solution. L'EDTA se lie aux ions (\( Ca^{2+} \)) dans l'eau. Le (\( NaOH \)) est utilisé pour augmenter le pH de la solution car l'EDTA se lie mieux au calcium dans des conditions légèrement basiques. L'indicateur coloré change de couleur en fonction de la quantité de (\( Ca^{2+} \)) libre dans la solution. L’équation de complexation de cette réaction est la suivante.

\[ Ca^{2+}_{~~(aq)}+EDTA^{4-}_{~~(aq)}\rightarrow CaEDTA^{2-}_{~~(aq)} \]



\section*{Manipulations}

\paragraph{}La procédure expérimentale a d’abord consisté à mesurer la concentration des ions \(Ca^{2+}\) libres dans l’eau du puits à l’aide de deux titrages complexométriques successifs. Dans chaque essai, une solution de \(NaOH\) et l’indicateur BBR ont été ajoutés à un volume donné d’échantillon, puis le mélange a été titré avec une solution d’EDTA jusqu’à l’obtention d’une coloration bleu pâle, correspondant à la complexation complète des ions \(Ca^{2+}\).

\paragraph{}Dans un second temps, un troisième échantillon d’eau de puits a été traité par ajout d’un excès de carbonate de sodium (\(Na_2CO_3\)) afin de précipiter les ions \(Ca^{2+}\) sous forme de carbonate de calcium (\(CaCO_3\)). Le précipité formé a été séparé par filtration, séché puis pesé afin de déterminer la quantité de solide obtenue. Le filtrat issu de cette étape a ensuite été titré de nouveau à l’EDTA pour mesurer la concentration résiduelle en ions \(Ca^{2+}\) en solution \parencite{piotte_chm131_2025}.


\section*{Résultats}
\subsection*{Partie 1: Titrages}
\begin{figure}[H]
    \centering
    \caption{\label{fig:figure1}Mesures du titrage 1}
    \small
    \begin{tabu} to \textwidth {
        | >{\centering\arraybackslash}m{2.5cm}
        | >{\centering\arraybackslash}m{2.5cm}
        | >{\centering\arraybackslash}m{2.5cm}
        | >{\centering\arraybackslash}m{2.25cm}
        | >{\centering\arraybackslash}m{2.5cm}
        | >{\centering\arraybackslash}m{2.25cm} |
        }
        \hline
        Grandeur                               & Mesure & Incertitude instrumentale & Incertitude expérimentale & Incertitude globale & Unité de mesure \\
        \hline
        Volume de l'échantillon d'eau de puits & 10     & $\pm 0,1$                 & $\pm 0,1$                 & $\pm 0,2$           & $\mathrm{mL}$   \\
        \hline
        Volume de titrant (EDTA)               & 7,6    & $\pm 0,05$                & $\pm 0,15$                & $\pm 0,2$           & $\mathrm{mL}$   \\
        \hline
    \end{tabu}
\end{figure}


\begin{figure}[H]
    \centering
    \caption{\label{fig:figure2}Mesures du titrage 2}
    \small
    \begin{tabu} to \textwidth {
        | >{\centering\arraybackslash}m{2.5cm}
        | >{\centering\arraybackslash}m{2.5cm}
        | >{\centering\arraybackslash}m{2.5cm}
        | >{\centering\arraybackslash}m{2.25cm}
        | >{\centering\arraybackslash}m{2.5cm}
        | >{\centering\arraybackslash}m{2.25cm} |
        }
        \hline
        Grandeur                               & Mesure & Incertitude instrumentale & Incertitude expérimentale & Incertitude globale & Unité de mesure \\
        \hline
        Volume de l'échantillon d'eau de puits & 10     & $\pm 0,1$                 & $\pm 0,1$                 & $\pm 0,2$           & $\mathrm{mL}$   \\
        \hline
        Volume de titrant (EDTA)               & 7,8    & $\pm 0,05$                & $\pm 0,15$                & $\pm 0,2$           & $\mathrm{mL}$   \\
        \hline
    \end{tabu}
\end{figure}

\subsection*{Parties 2 et 3 : Précipitation et gravimétrie}


\begin{figure}[H]
    \centering
    \caption{\label{fig:figure3}Mesures de la précipitation}
    \small
    \begin{tabu} to \textwidth {
        | >{\centering\arraybackslash}m{2.5cm}
        | >{\centering\arraybackslash}m{2.5cm}
        | >{\centering\arraybackslash}m{2.5cm}
        | >{\centering\arraybackslash}m{2.25cm}
        | >{\centering\arraybackslash}m{2.5cm}
        | >{\centering\arraybackslash}m{2.25cm} |
        }
        \hline
        Grandeur                                       & Mesure & Incertitude instrumentale & Incertitude expérimentale & Incertitude globale & Unité de mesure \\
        \hline
        Volume de l'échantillon d'eau de puits         & 200    & $\pm 1$                   & $\pm 1$                   & $\pm 2$             & $\mathrm{mL}$   \\
        \hline
        Masse de $Na_2CO_3$ pesée                      & 0,252  & $\pm 0,001$               & $\pm 0,002$               & $\pm 0,003$         & $\mathrm{g}$    \\
        \hline
        Masse du filtre seul                           & 0,098  & $\pm 0,001$               & $\pm 0,001$               & $\pm 0,002$         & $\mathrm{g}$    \\
        \hline
        Masse du filtre et du précipité après séchage  & 0,293  & $\pm 0,001$               & $\pm 0,001$               & $\pm 0,002$         & $\mathrm{g}$    \\
        \hline
        Volume de filtrat prélevé                      & 50     & $\pm 0,5$                 & $\pm 0,5$                 & $\pm 1$             & $\mathrm{mL}$   \\
        \hline
        Volume du titrant (EDTA) du titrage du filtrat & 0,9    & $\pm 0,05$                & $\pm 0,15$                & $\pm 0,2$           & $\mathrm{mL}$   \\
        \hline
    \end{tabu}
\end{figure}

\clearpage
\subsection*{Calculs}
\begin{enumerate}
    \item $(7,7\pm 0,2)\,\mathrm{mL}$
    \item $(7,7\pm 0,4)\times 10^{-3}\,\mathrm{mol\cdot L^{-1}}$
    \item \begin{enumerate}
              \item $(1,54 \pm 0,09)\times 10^{-3}\,\mathrm{mol}$
              \item $(2,38 \pm 0,03)\times 10^{-3}\,\mathrm{mol}$
              \item L'ion $\mathrm{Ca^{2+}}$ est limitant, car sa quantit\'e ($1,54 \times 10^{-3}\,\mathrm{mol}$) est plus petite que celle de $\mathrm{CO_3^{2-}}$ ($2,38 \times 10^{-3}\,\mathrm{mol}$).
              \item $(55 \pm 9)\,\%$
          \end{enumerate}
    \item $(1,8 \pm 0,4) \times 10^{-4}\,\mathrm{mol\cdot L^{-1}}$
    \item \begin{enumerate}
              \item $(1,54 \pm 0,09) \times 10^{-3}\,\mathrm{mol}$
              \item $(3,6 \pm 0,5) \times 10^{-5}\,\mathrm{mol}$
              \item $(97,7 \pm 0,6)\,\%$
          \end{enumerate}
    \item \begin{enumerate}
              \item $(1,95 \pm 0,03) \times 10^{-3}\,\mathrm{mol}$
              \item $(1,99 \pm 0,04) \times 10^{-3}\,\mathrm{mol}$
              \item $(129 \pm 7)\,\%$
          \end{enumerate}
\end{enumerate}
\subsubsection*{Exemple de calculs Q.5}
\begin{figure}[H]
    \centering
    \begin{gather}
        1,54 \times 10^{-3}~\text{mol}~Ca^{2+} \text{ dans l'échantillon initial (Q5a)} \\
        \frac{200~\text{mL (volume du filtrat)}}{50~\text{mL (volume d'échantillon)}} = 4 \\
        (9 \times 10^{-6}~\text{mol}) \times 4 = 3,6 \times 10^{-5}~\text{mol}~Ca^{2+} \text{ dans 200 mL (Q5b)}\\
        \text{Conversion de } Ca^{2+} = \frac{n_{\text{initial}} - n_{\text{restant}}}{n_{\text{initial}}} \times 100 \\
        = \frac{1,54 \times 10^{-3} - 3,6 \times 10^{-5}}{1,54 \times 10^{-3}} \times 100 \\
        = 97,7\% \approx (97,7 \pm 0,6)\,\%
    \end{gather}

\end{figure}
\clearpage

\section*{Interprétation des résultats}

\subsection*{Qualité des mesures}

\paragraph{}Le pourcentage de recouvrement obtenu pour les ions calcium est de \((129 \pm 7)\,\%\). Même en tenant compte de l’incertitude, la borne inférieure de l’intervalle (\(122\,\%\)) demeure nettement supérieure à \(100\,\%\). Un tel résultat ne peut pas être expliqué uniquement par les incertitudes de mesure normales et indique la présence probable de biais systématiques ou d’effets non pris en compte dans le modèle de calcul.

\paragraph{}Deux hypothèses principales peuvent être formulées pour expliquer ce recouvrement supérieur à \(100\,\%\) : la présence d’eau résiduelle dans le précipité lors de la pesée et la co-précipitation éventuelle d’autres cations, comme les ions magnésium, avec le carbonate de calcium.

\subsubsection*{Filtre partiellement humide lors de la pesée}

\paragraph{}Le précipité de \(CaCO_3\) est récupéré sur un filtre, puis séché au four à micro-ondes avant la pesée finale. Si le filtre et le précipité ne sont pas complètement secs, une petite quantité d’eau peut rester emprisonnée dans le solide ou absorbée par le papier. La masse du filtre et du précipité après séchage est de \(0,293~\mathrm{g}\), alors que la masse du filtre seul est de \(0,098~\mathrm{g}\), ce qui conduit à une masse de précipité d’environ \(0,195~\mathrm{g}\).

\paragraph{}Une erreur de seulement \(0,01\) à \(0,02~\mathrm{g}\) liée à de l’eau résiduelle représente déjà une surestimation de l’ordre de \(5\) à \(10\,\%\) de la masse du précipité (\(0,01/0,195 \approx 5\,\%\) et \(0,02/0,195 \approx 10\,\%\)). Une telle contribution suffit à expliquer une partie importante de l’écart observé entre \(100\,\%\) et les \(129\,\%\) mesurés pour le recouvrement. Un séchage plus long, l’utilisation d’un four conventionnel ou une vérification de la masse après refroidissement auraient permis de réduire cet effet.

\subsubsection*{Présence possible d’ions magnésium en solution}

\paragraph{}L’eau souterraine à l’origine de la dureté contient habituellement, en plus des ions \(Ca^{2+}\), une certaine quantité d’ions magnésium \(Mg^{2+}\). Lors de l’ajout de carbonate de sodium, ces ions peuvent également précipiter sous forme de carbonate de magnésium (\(MgCO_3\)), qui est peu soluble. Le protocole de calcul suppose toutefois que l’ensemble du précipité correspond uniquement à du \(CaCO_3\).

\paragraph{}Si la concentration en \(Mg^{2+}\) représente, par exemple, une fraction de l’ordre de \(20\) à \(30\,\%\) de la concentration en \(Ca^{2+}\), la masse totale du solide recueilli pourrait être surestimée d’une valeur comparable. Dans ce cas, la quantité de matière attribuée au calcium à partir de la masse du précipité serait plus élevée que la quantité réellement présente au départ, ce qui conduit naturellement à un recouvrement apparent supérieur à \(100\,\%\). Cette hypothèse est donc compatible, en ordre de grandeur, avec la valeur expérimentale de \((129 \pm 7)\,\%\).

\subsection*{Rôle et quantité de Na\texorpdfstring{$_2$}{\_2}CO\texorpdfstring{$_3$}{\_3}}

\paragraph{}L’analyse stœchiométrique montre que, pour le test en bécher, la quantité de matière en ions carbonate fournie par le \(Na_2CO_3\) est plus élevée que la quantité de matière en ions calcium présente dans l’échantillon. L’ion \(Ca^{2+}\) est donc le réactif limitant, tandis que les ions carbonate sont en excès. Ce choix garantit que la disponibilité du carbonate ne limite pas la précipitation et que les ions responsables de la dureté peuvent, en principe, être presque entièrement transformés en \(CaCO_3\).

\paragraph{}Le pourcentage de conversion calculé pour les ions \(Ca^{2+}\) à partir de la concentration résiduelle après traitement est de \((97,7 \pm 0,6)\,\%\). Ce résultat indique que, malgré les biais mis en évidence par l’analyse du recouvrement, le traitement par carbonate de sodium a permis d’éliminer la grande majorité des ions calcium de la solution. Dans une perspective de mise à l’échelle, il serait pertinent de conserver un léger excès de \(Na_2CO_3\) pour assurer une bonne conversion, tout en optimisant les conditions d’agitation, de temps de réaction et de séparation du précipité afin de limiter les pertes et les surestimations de masse.

\subsection*{Évaluation du prix pour traiter 800~hL de bière}

\paragraph{}Pour estimer le coût du traitement de l’eau nécessaire à la production de \(800~\mathrm{hL}\) de bière, un intervalle de volume total d’eau a d’abord été établi. Selon une source spécialisée en brassage amateur, il faut entre \(7\) et \(10~\mathrm{L}\) d’eau pour produire \(1~\mathrm{L}\) de bière \parencite{noauthor_autobrasseur_nodate}. En appliquant ces valeurs, la préparation de \(800~\mathrm{hL}\) (soit \(80\,000~\mathrm{L}\) de bière) nécessite donc entre \(560\,000~\mathrm{L}\) et \(800\,000~\mathrm{L}\) d’eau.

\paragraph{}Le protocole expérimental utilise une masse de \(0,252~\mathrm{g}\) de \(Na_2CO_3\) pour traiter \(200~\mathrm{mL}\) d’eau, ce qui correspond à environ \(1,26~\mathrm{g}\) de \(Na_2CO_3\) par litre d’eau traitée. En appliquant ce dosage aux volumes estimés, la masse totale de carbonate de sodium requise se situe entre environ \(705,6~\mathrm{kg}\) (pour \(560\,000~\mathrm{L}\)) et \(1008~\mathrm{kg}\) (pour \(800\,000~\mathrm{L}\)).

\paragraph{}En considérant un prix d’environ \(1{,}24~\mathrm{CAD/kg}\) pour des commandes en vrac de qualité appropriée pour un usage agro-alimentaire, le coût du \(Na_2CO_3\) nécessaire au traitement de l’eau serait donc compris approximativement entre \(875~\mathrm{CAD}\) et \(1250~\mathrm{CAD}\) \parencite{noauthor_sodium_2025}. Ce coût demeure modeste par rapport à l’échelle d’un projet de micro-brasserie et ne constitue pas un frein majeur à l’implantation du procédé d’adoucissement.

\section*{Conclusion}

\paragraph{}Les résultats obtenus montrent que le traitement par précipitation au carbonate de sodium permet de réduire significativement la concentration en ions \(Ca^{2+}\) dans l’eau de puits de Loublonnière. Après traitement, la concentration résiduelle en calcium est de l’ordre de \(1{,}8 \times 10^{-4}\,\mathrm{mol\cdot L^{-1}}\), soit environ \(0{,}18~\mathrm{mmol\cdot L^{-1}}\), ce qui est inférieur à la valeur cible de \(0{,}50~\mathrm{mmol\cdot L^{-1}}\) généralement retenue pour limiter la dureté. Sur le plan strictement chimique, le procédé testé atteint donc l’objectif de traitement fixé pour une utilisation dans un procédé de micro-brasserie.

\paragraph{}Cependant, l’analyse du pourcentage de recouvrement met en évidence des biais systématiques importants, notamment liés au séchage du précipité et à la possible co-précipitation d’autres ions. De plus, la mise en œuvre industrielle d’un procédé par précipitation et filtration impose une gestion rigoureuse des étapes d’agitation, de séparation solide-liquide et d’élimination des boues, ce qui peut complexifier l’exploitation à grande échelle.

\paragraph{}Pour ces raisons, l’implantation de la brasserie à Loublonnière apparaît envisageable, mais l’utilisation d’un système d’adoucissement basé sur des résines échangeuses d’ions, de type \(2Na^+\), est recommandée pour le traitement de l’eau d’alimentation. Ce type de filtre permet de fixer les ions \(Ca^{2+}\) et \(Mg^{2+}\) de manière continue, avec une capacité opérationnelle élevée et une intégration plus simple dans une chaîne de production industrielle \parencite{purolite_industrial_nodate}. L’inconvénient principal réside dans la nécessité de régénérer périodiquement les résines et de gérer les effluents de régénération, ce qui ajoute une étape de maintenance supplémentaire \parencite{purolite_purolite_nodate}. Malgré ces contraintes, cette solution offre une meilleure maîtrise de la dureté de l’eau et une plus grande robustesse pour un usage industriel que le procédé par précipitation testé au laboratoire.

%---------------------------------------------------------------
% BIBLIOGRAPHIE
%---------------------------------------------------------------
\newpage
\nocite{chat}
\nocite{google}
\printbibliography[title={Bibliographie}]
\clearpage
\section*{Annexe}

\paragraph{}L’incertitude instrumentale a été déterminée en fonction des caractéristiques des appareils utilisés pour effectuer les mesures. Pour les mesures de masse obtenues à l’aide de la balance numérique, l’incertitude a été fixée à la moitié de la plus petite division affichée. Pour les volumes, les incertitudes ont été établies à partir des graduations visibles sur les instruments de verrerie (béchers, cylindres gradués, burette), en tenant compte de la précision de lecture réaliste.

\paragraph{}L’incertitude expérimentale a été estimée à partir du jugement porté sur la reproductibilité des manipulations et sur les principales sources de variabilité observées. Des écarts ont notamment pu apparaître dans le volume de filtrat récupéré, dans la masse du précipité après séchage ou encore dans la détermination du point de virage lors du titrage, en raison de la perception de la couleur et de l’éclairage. L’incertitude globale pour chaque mesure résulte de la combinaison de ces contributions instrumentales et expérimentales.

\end{document}
