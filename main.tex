%=================================================================
% Modèle de devoir - Version française
%=================================================================

\documentclass[12pt,a4paper]{article}

%---------------------------------------------------------------
% PACKAGES
%---------------------------------------------------------------
\usepackage[T1]{fontenc}              % Encodage correct
\usepackage[utf8]{inputenc}           % Support UTF-8
\usepackage[french]{babel}            % Langue française
\usepackage{geometry}                 % Marges
\geometry{margin=2.0cm}
\usepackage{amsmath,amssymb,amsthm}   % Math
\usepackage{graphicx}                 % Images
\usepackage{array}                    % Tableaux avancés
\usepackage{tikz}                     % K-maps, circuits logiques
\usetikzlibrary{matrix, positioning}  % Bibliothèques TikZ nécessaires
\usepackage{caption}                  % Légendes personnalisées
\usepackage{multirow}                 % Cellules multi-lignes
\usepackage{xcolor}                   % Couleurs
\usepackage{fancyhdr}                 % En-têtes / pieds de page
\usepackage{hyperref}                 % Liens hypertextes
\usepackage{float}                    % Placement précis
\usepackage{datetime}                 % Formatage de dates
\usepackage{ctable}
\usetikzlibrary{calc}
\usepackage{ragged2e}
\usepackage{csquotes}                 % Recommandé avec biblatex
\usepackage{newtxtext,newtxmath}
\usepackage{setspace}
\usepackage{tabu}

% --- BIBLIOGRAPHIE (style APA ÉTS) ---
\usepackage[style=apa,backend=biber,sorting=nyt]{biblatex}
\setlength\bibitemsep{0.8\baselineskip}
\DeclareLanguageMapping{french}{french-apa}
\addglobalbib{sample.bib}

%---------------------------------------------------------------
% INFORMATIONS DU COURS ET DU GROUPE
%---------------------------------------------------------------
\newcommand{\codeclasse}{CHM131-}                     % Code du cours
\newcommand{\nomclasse}{Chimie et matériaux}          % Nom du cours
\newcommand{\titredevoir}{Laboratoire: Adoucissement par précipitation}
\newcommand{\groupe}{15}
\newcommand{\enseignant}{Marc Fraser}
\newcommand{\datedepot}{21 octobre 2025}
\newcommand{\Session}{A25}
\newcommand{\auteurs}{
    Ourania Voyatzis~(VOYO78260401)\\
    Kian Chowanietz~(CHOK86330401)\\
    Gubens Romulus~(ROMG77300401)\\
    Loïc Lompo~(LOMY76310501)
}

%---------------------------------------------------------------
% DÉBUT DU DOCUMENT
%---------------------------------------------------------------
\begin{document}
\onehalfspacing

\begin{titlepage}
\RaggedLeft\auteurs\\
    \vspace*{\fill}
    \centering{
    \textbf{\titredevoir}\\[0.5em]
    \vspace*{\fill} 
    \textbf{\nomclasse\ (\codeclasse\groupe)}\\[0.5em]
    \textbf{Remis à: }\enseignant \\[1em]
    \vspace*{\fill}

    \textbf{Date: } \datedepot\\
    \textbf{Date de remise:}\text{ 24 novembre 2025}
    }
\end{titlepage}



%---------------------------------------------------------------
% CONTENU
%---------------------------------------------------------------
\section*{Introduction}

\paragraph{}Les sources d'eau dure peuvent poser de nombreux problèmes aux procédés industriels ainsi qu'à certains types d'instruments et aux canalisations. La principale solution à ce problème courant consiste à utiliser différentes méthodes d'adoucissement de l'eau. L'eau dure est généralement, et dans ce cas précis, une eau contenant une forte concentration d'ions calcium (\(Ca^{2+}\)) dissous. Bien qu'il puisse également s'agir d'un autre ion ou d'une combinaison d'ions, l'objectif final de l'adoucissement de l'eau peut, dans tous les cas, être atteint par des moyens chimiques et/ou mécaniques. Dans le cadre de ce laboratoire, l'objectif est d'adoucir l'eau pour faire de la bière en éliminant les ions calcium afin qu'elle soit utilisable dans ce procédé industriel. Pour ce faire, la concentration de calcium est mesurée, précipitée hors de la solution et filtrée mécaniquement afin d'obtenir le produit final : de l'eau adoucie.

\section*{Principe}

\paragraph{}Les tests préliminaires réalisés sur l’eau souterraine de Loublonnière ont montré qu’elle présente des qualités organoleptiques et chimiques favorables à la fabrication de la bière, à l’exception d’une dureté trop élevée \parencite{piotte_chm131_2025}. Une eau de consommation est considérée de bonne qualité lorsqu’elle possède un goût neutre, sans arrière-goût métallique, et qu’elle est incolore, limpide et sans odeur perceptible \parencite{mediaterre_importance_2013}. La présence de certains ions peut toutefois influencer positivement le goût final de la bière : par exemple, une eau contenant une concentration de chlorure supérieure à celle de sulfate peut \textit{« rehausser la rondeur et la douceur du malt »} \parencite{giovanisci_beginners_2022}.

\paragraph{}La dureté de l’eau est principalement causée par la présence d’ions calcium (\(Ca^{2+}\)) et magnésium (\(Mg^{2+}\)) dissous. Une concentration trop élevée en ces ions peut nuire à certains procédés industriels et altérer la saveur des boissons produites avec cette eau. \textit{« Les eaux souterraines qui ont été en contact avec des roches poreuses contenant des dépôts de minéraux tels que le calcaire ou la dolomie seront très dures »} en raison de la dissolution et de la dissociation de ces minéraux dans l'eau \parencite{mcvean_is_2019}. Cependant, cette dureté peut également être bénéfique dans une certaine mesure pour le brassage de la bière si la source est la gypse.
	\subparagraph{}\textit{« Les effets positifs du gypse sont de réduire le pH du moût, d'améliorer l'efficacité de l'extraction du malt grâce à une activité amylolytique accrue, de donner un pouvoir tampon au moût, d'équilibrer l'arôme du houblon pour les bières fortement houblonnées, d'améliorer la clarté du moût et d'éliminer les phosphates et les protéines dans les résidus du moût. Toutefois, ce dernier effet d'élimination des ions phosphates (sous forme de phosphate de calcium insoluble) peut être exagéré et nuire à la fermentation si le moût est trop pauvre en ions phosphates. »} \parencite[p.~562]{oliver_oxford_2013}
\paragraph{} L'équation chimique de la dissolution du gypse dans l'eau est la suivante.
\[ CaSO_{4~~(s)} + 2H_2O \rightarrow Ca^{2+}_{~~(aq)} + SO^{2-}_{4~~(aq)} \]

\paragraph{}Dans ce laboratoire, l’adoucissement de l’eau est effectué par précipitation chimique. Un réactif, le carbonate de sodium (\(Na_2CO_3\)), contenant des ions carbonate (\(CO_3^{2-}\)) est ajouté afin de former un précipité insoluble de carbonate de calcium, éliminant ainsi une partie des ions responsables de la dureté. L’équation ionique simplifiée de cette réaction est la suivante :

\[
\mathrm{Ca^{2+}_{~~(aq)} + CO^{2-}_{3~~(aq)} \rightarrow CaCO_{3~(s)}}
\]

\paragraph{}Le carbonate de calcium précipité est ensuite séparé mécaniquement par filtration, permettant d’obtenir une eau adoucie dont la concentration en ions calcium est réduite.


\paragraph{}Dans ce laboratoire, la quantité d'ions calcium libres dans l'eau du puits de Loublonnière a été mesurée par titrage de l'EDTA dans la solution. L'EDTA se lie aux ions (\( Ca^{2+} \)) dans l'eau. Le (\( NaOH \)) est utilisé pour augmenter le pH de la solution car l'EDTA se lie mieux au calcium dans des conditions légèrement basiques. L'indicateur coloré change de couleur en fonction de la quantité de (\( Ca^{2+} \)) libre dans la solution. L'équation chimique de complexification pour cette réaction est la suivante.  

\[ Ca^{2+}_{~~(aq)}+EDTA^{4−}_{~~(aq)}\rightarrow CaEDTA^{2−}_{~~(aq)} \]



\section*{Manipulations}
\paragraph{}La procédure consistait d'abord à mesurer la concentration des ions (\( Ca^{2+} \)) libres dans l'eau du puits dans deux essais différents. Pour ce faire, du (\( NaOH \)) et de l'indicateur coloré BBR as était ajouter à l'eau. Nous avons ensuite effectué un titrage à l'EDTA jusqu'à ce que les deux échantillons deviennent bleu pale, lorsque tout le calcium a été lié à l'EDTA et qu'il ne reste plus d'ions libres.
Ensuite, la réaction de précipitation avec (\(Na_2CO_3\)) a été effectuée avec un troisième échantillon, qui a ensuite été filtré et dont la quantité de précipité a été mesurée. \parencite{piotte_chm131_2025}


\section*{Résultats}
\subsection*{Partie 1: Titrages}
\begin{figure}[H]
	\centering
	\caption{\label{fig:figure1}Mesures du titrage 1}
	\small
	\begin{tabu} to \textwidth { 
			| >{\centering\arraybackslash}m{2.5cm} 
			| >{\centering\arraybackslash}m{2.5cm} 
			| >{\centering\arraybackslash}m{2.5cm} 
			| >{\centering\arraybackslash}m{2.25cm} 
			| >{\centering\arraybackslash}m{2.5cm} 
			| >{\centering\arraybackslash}m{2.25cm} | 
		}
		\hline
		Grandeur & Mesure & Incertitude instrumentale & Incertitude expérimentale & Incertitude globale & Unité de mesure \\
		\hline
		Volume de l'échantillon d'eau de puits & 10 & $\pm 0,1$ & $\pm 0,1$ & $\pm 0,2$ & mL \\
		\hline
		Volume de titrant (EDTA) & 7.6 & $\pm 0,05$ & $\pm 0,15$ & $\pm 0,2$ & mL \\
		\hline
	\end{tabu}
\end{figure}


\begin{figure}[H]
	\centering
	\caption{\label{fig:figure2}Mesures du titrage 2}
	\small
	\begin{tabu} to \textwidth { 
			| >{\centering\arraybackslash}m{2.5cm} 
			| >{\centering\arraybackslash}m{2.5cm} 
			| >{\centering\arraybackslash}m{2.5cm} 
			| >{\centering\arraybackslash}m{2.25cm} 
			| >{\centering\arraybackslash}m{2.5cm} 
			| >{\centering\arraybackslash}m{2.25cm} | 
		}
		\hline
		Grandeur & Mesure & Incertitude instrumentale & Incertitude expérimentale & Incertitude globale & Unité de mesure \\
		\hline
		Volume de l'échantillon d'eau de puits & 10 & $\pm 0,1$ & $\pm 0,1$ & $\pm 0,2$ & mL \\
		\hline
		Volume de titrant (EDTA) & 7.8 & $\pm 0,05$ & $\pm 0,15$ & $\pm 0,2$ & mL \\
		\hline
	\end{tabu}
\end{figure}

\subsection*{Parties 2 et 3 : Précipitation et gravimétrie}


\begin{figure}[H]
	\centering
	\caption{\label{fig:figure3}Mesures du précipitation}
	\small
	\begin{tabu} to \textwidth { 
			| >{\centering\arraybackslash}m{2.5cm} 
			| >{\centering\arraybackslash}m{2.5cm} 
			| >{\centering\arraybackslash}m{2.5cm} 
			| >{\centering\arraybackslash}m{2.25cm} 
			| >{\centering\arraybackslash}m{2.5cm} 
			| >{\centering\arraybackslash}m{2.25cm} | 
		}
		\hline
		Grandeur & Mesure & Incertitude instrumentale & Incertitude expérimentale & Incertitude globale & Unité de mesure \\
		\hline
		Volume de l'échantillon d'eau de puits & 200 & $\pm 1 $ & $\pm 1$ & $\pm 2$ & mL \\
		\hline
		Masse de $Na_2CO_3$ pessée & 0,252 & $\pm$ 0,001 & $\pm$ 0,002 & $\pm$ 0,003 & g \\
		\hline
		Masse du filtre seul & 0,098 & $\pm$0,001 & $\pm$0,001 & $\pm$0,002 & g \\
		\hline
		Masse du filtre et du précipité après séchage  & 0,293 & $\pm$ 0,001 & $\pm$ 0,001 & $\pm$ 0,002 & g \\
		\hline
		Volume de filtrat prélevé  & 50 & $\pm$ 0,5 & $\pm$ 0,5 & $\pm$ 1 & mL \\
		\hline
		Volume du titrant (EDTA) du titrage du filtrat & 0,9 & $\pm$ 0,05 & $\pm$ 0,15 & $\pm$ 2 & mL \\
		\hline
	\end{tabu}
\end{figure}

\subsection*{Calculations}
\begin{enumerate}
	\item 7,7 $\pm$ 0,2 mL
	\item 7,7 x $10^-3$
\end{enumerate}

%---------------------------------------------------------------
% BIBLIOGRAPHIE
%---------------------------------------------------------------
\newpage
\printbibliography[title={Bibliographie}]

\end{document}
